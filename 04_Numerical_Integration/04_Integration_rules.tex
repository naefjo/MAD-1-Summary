\subsection{Integration Rules} \label{ss:intrules}
    \begin{itemize}
        \item \textbf{Rectangle/Midpoint Rule:} 
            \begin{equation*}
                \colorboxed{red}{I_{R_i} = f(x_i)\Delta_i} \
                \colorboxed{red}{I_{M_i} = f\left(\frac{x_i + x_{i+1}}{2}\right)\Delta_i}
            \end{equation*}
            
        \item \textbf{Trapezoidal Rule:}
            \begin{equation*}
                \colorboxed{red}{I_{T_i} = \frac{f(x_i) + f(x_{i+1})}{2}\Delta_i}
            \end{equation*}
        \item \textbf{Simpson's Rule:}
            \begin{equation*}
                \colorboxed{red}{I_{S_i} = \frac{f(x_i) + 4 f\left(\frac{x_i + x_{i+1}}{2}\right) + f(x_{i+1})}{6}\Delta_i}
            \end{equation*}
    \end{itemize}
    
    \subsubsection{Total Integrals}\label{sssec:totalint}
        \textbf{Rectangle/Midpoint Rule:} 
                \begin{equation*}
                    \colorboxed{red}{I \approx \Delta_i \sum_{i=0}^{N-1}f(x_i)} \
                    \colorboxed{red}{I \approx \Delta_i \sum_{i=0}^{N-1}f\left(\frac{x_i + x_{i+1}}{2}\right)}
                \end{equation*}
        % \textbf{Midpoint Rule:} 
        %     \begin{equation*}
        %         \colorboxed{red}{I \approx \Delta_i \sum_{i=0}^{N-1}f\left(\frac{x_i + x_{i+1}}{2}\right)}
        %     \end{equation*}
        \textbf{Trapezoidal Rule:} 
            \begin{equation*}
                \colorboxed{red}{I \approx \frac{\Delta_i}{2}\left(f(x_0) + 2\sum_{i=1}^{N-1} f(x_i) + f(x_N)\right)}
            \end{equation*} 
        \textbf{Simpson's Rule:}
            \begin{equation*}
                \colorboxed{red}{\frac{\Delta_i}{3}\left( f(x_0) + 4\sum_{\substack{i = 1 \\ i = \textrm{odd}}}^{N-1} f(x_i) + 2\sum_{\substack{i = 2 \\ i = \textrm{even}}}^{N-2}f(x_i) + f(x_N)\right)}
            \end{equation*}

\subsection{Newton-Cotes Formulas}
    General way to derive quadrature rules. We use $M+1$ equidistant points in $[x_i, x_{i+1}]$ $(x_k = x_i +k\cdot h, k =0,\dots, M)$ and Lagrange interpolation. The lagrange interpolant trought $(x_k, f(x_k))$ is given by
    \begin{gather*}
        \colorboxed{red}{I_i \approx \Delta_i\sum_{k=0}^M C_k^M f(x_k)}\
        \colorboxed{red}{C_k^M = \frac{1}{\Delta_i}\int_{x_i}^{x_{i+1}}l_k^M(x)dx}\\
        l_k^M(x) = \prod_{\substack{i=0 \\ i \neq k}}^M \frac{(x - x_i)}{(x_k - x_i)}
    \end{gather*}
    
    \textbf{Properties of $C_k^M$:}
        \begin{itemize}
            \item $\sum_{k=0}{M} C_k^M= 1$
            \item $C_k^M = C_{M-k}^{M}$
        \end{itemize}

\subsection{Error Analysis}
    Find an upper bound for our error $E_{\textrm{rule},i} = I_i - I_{\textrm{rule},i}$
    
    \textbf{Midpoint Rule:} Taylor series around $x_{i+1/2}$ yields:
        \begin{equation*}
            \colorboxed{red}{E_{M_i} = \frac{1}{24}f''(x_{i+1/2})\Delta_i^3+ O(\Delta_i^5) + \dots}
        \end{equation*}
       One interval midpoint rule is \textbf{third order accurate}.
        
    \textbf{Trapezoidal Rule:} 
        \begin{equation*}
            \colorboxed{red}{E_{T_i} = -\frac{1}{12}f''(x_{i+1/2})\Delta_i^3 + O(\Delta_i^5)}
        \end{equation*}
    \textbf{Simpson's Rule:} $I_{S_i} = \frac{2}{3}I_{M_i} + \frac{1}{3}I_{T_i}$
        \begin{equation*}
            \colorboxed{red}{E_{S_i} = O(\Delta_i^5) + \dots} 
        \end{equation*}
        
        \textbf{Order reduces by one for whole domain} (e.g. 2\textsuperscript{nd} for midpoint/trap; 4\textsuperscript{th} for Simpson).
        % These are all for \textbf{ONE INTERVAL!!} over the whole domain with constant spacing the order is reduced by one order (ie second order for Midpoint and trapezoidal and fourth order for Simpson).