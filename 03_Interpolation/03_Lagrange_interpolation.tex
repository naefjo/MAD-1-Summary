\subsection{Lagrange Interpolation}
    We want a base polynomial which satisfies $l_i(x_j) = \delta_{ij}$. Create a function of order $N-1$ which perfectly fits $N$ data points.
    \begin{equation*}
        \underbrace{\colorboxed{red}{\ell_k(x) = \prod_{\substack{i=1 \\ i \neq k}}^N \frac{x-x_i}{x_k-x_i}
        }}_{\textnormal{Lagrange Functions}}
        \,
        \underbrace{\colorboxed{red}{
        f(x)=\sum_{k=1}^N y_k \ell_k(x)
        }}_{\textnormal{Lagrange Interpolator}}
    \end{equation*}
    
   
    
    \subsubsection{LSQ vs Interpolation}
        \textbf{Noise:} LSQ robust to noise. Interpolation passes exactly through $(x_i,y_i)\forall i$.
        \textbf{Note:} Lagrange function is dependant ond $x_i$ only.
        
        \textbf{Data points:} Terms of higher order than order of data cancel out in the Lagrange interpolator. I.e. if data linear but 5 data points, lagrange will still be linear. \textbf{LSQ also robust to few data points.}
        
        \textbf{Stability:} Hard to approx. loc. behaviour if interpolator used as global function. Oscillatory behaviour at the endpoints (higher order). Fluctuations in one region may affect the function over the whole domain.
        % LSQ better than Interpolation since Interpolation will pass exactly through the points and take noise into account whereas LSQ is robust to noise. 